\chapter{Literature Survey}

\section{The PUF concept}
The simplest sentence to describe PUF is "A PUF is an object's fingerprint" \cite{Reference4}. The fingerprint can represent a specific human in the world, such as the PUF can represent
an object. The fingerprint is inherently created when people were born, and so does PUF, which inherently exists in an object according to unique manufacturing random variation \cite{Reference4}.
With the representation and inherent property, the fingerprint and the PUF is said to be unclonable since it is impossible to control and predict a human's fingerprint. This is an important concept for PUF. \par

This intrinsic property can be extracted from a chip that has PUF circuit that existed inside \cite{Reference2}. The way PUF works is by entering a certain length of bits(so-called challenge) into the PUF, and it will
generate another specific length of bits(so-called response). According to the property of PUF that was discussed above, it is impossible to find two different PUF that will produce the same response when entering the same challenge(See Figure \ref{fig:figure1}).
\begin{figure}[htp]
\centering
\includegraphics[width=8cm]{figures/figure1.jpg}
\caption{Different PUF that generate a different response when input same challenge}
\label{fig:figure1}
\end{figure}

\section{Weak and strong PUF}
PUF can be classified into two categories, weak and strong PUF according to the strength of PUF. The strength of PUF indicates the number of challenge-response pairs(so-called CRPs) that can be generated 
from the PUF \cite{Reference1}. The higher numbers of CRPs can a PUF generate, the better strength it has. Generally, if increasing the size of the PUF leads to a linear increase in the number of CRPs, it is considered weak PUF. 
On the other hand, if increasing the size of the PUF leads to an exponential increase in the number of CRPs, it is considered strong PUF.\par

For the weak PUF, it represents the PUF that has a smaller set of CRPs. While it is impossible to 
create a clone of PUF, but with a small set of CRPs, this will allow an attacker to record all the CRPs when the attacker has physical access to PUF \cite{Reference1}. With the knowledge of CRPs, attackers can easily provide the corresponding
response to challenge as like they have a clone(See Figure \ref{fig:figure2}). The weak PUF can be used for authentication and key storage. However, since weak PUF's CRPs can be fully accessed, ensuring having a secure environment and whether the original PUF is being evaluated is relatively important \cite{Reference1}.
\begin{figure}[htp]
    \centering
    \includegraphics[width=10cm]{figures/figure2.jpg}
    \caption{Attacker can perform same behaviour as Weak PUF when having full access to CRPs and not under secure environment}
    \label{fig:figure2}
    \end{figure}

For strong PUF, means the number of CRPs is significantly large that even attackers get access, having throughout knowledge of CRPs is impossible. While the number of CRPs is so large,
and the CRP is randomly selected in usage, the probability that the attacker has knowledge about the CRP currently using is small. In addition, each CRPs that is used once will 
be discarded(See Figure \ref{fig:figure3}) so even if attackers recorded certain CRPs, also called eavesdropped, they will not be able to put them into use. The strong PUF can also be used for authentication but do not need to protect CRPs
as serious as weak PUF.

\begin{figure}[htp]
    \centering
    \includegraphics[width=15cm]{figures/figure3.jpg}
    \caption{The attacker eavesdropped CRP that has been used can not successfully validate in next evaluation for strong PUF}
    \label{fig:figure3}
    \end{figure}

\section{Authentication}
One of the applications of PUF is authentication. As discussed in Chapter 1's introduction, PUF does not require huge computational power and are cost-effective, so it is suitable for many devices,
especially the resources-constraint devices. The PUF's authentication included two stages, the enrollment and the authentication stage. In the enrollment stage, the company possess the PUF, so
the company can connect the server to PUF and send lots of challenges along with recording the CRPs into the database \cite{Reference2} (See Figure \ref{fig:figure4}).

\begin{figure}[htp]
    \centering
    \includegraphics[width=8cm]{figures/figure4.jpg}
    \caption{Enrollment stage in PUF authentication}
    \label{fig:figure4}
    \end{figure}

After recording all the CRPs, the company can now implement PUF on electronic devices.
In the authentication stage, the server sent an arbitrary challenge to the devices that contain PUF while the device will return response. Afterwards, the server compares the response from the device with the database, 
if the challenge and response pair exists in the database, the device is valid \cite{Reference2} (See Figure \ref{fig:figure5}). A life example will be a banking card.

\begin{figure}[htp]
    \centering
    \includegraphics[width=8cm]{figures/figure5.jpg}
    \caption{Authentication stage in PUF authentication}
    \label{fig:figure5}
    \end{figure}

\section{Arbiter PUF and XOR arbiter PUF}
There are many different types of PUF such as arbiter PUF, ring oscillator PUF, lightweight PUF, etc. In this paper, arbiter PUF and its mutation will be introduced in detail. The general idea of
the arbiter PUF is comparing the transition speed for two electrical signals in the PUF's structure(See Figure \ref{fig:figure6}). The arbiter PUF's structure contains a number of 
multiplexers and an arbiter(mostly D flipflop), and two multiplexers will be combined into a switching box \cite{Reference3}. Look at Figure \ref{fig:figure6}, when entering a challenge bit, apply each challenge bit to a switching box, bit 1 indicate the upper and lower signal will switch 
while bit 0 indicate the two signals remain unchanged in each switching box. This will eventually form paths for the signals. Then the signals start transferring, the time arrived at the arbiter for two signals 
is different since each multiplexer and wire has a unique delay. The arbiter will determine which path is faster and based on that response a binary bit if the upper path is faster, the response is 1, 
otherwise, the response is 0. \par

The arbiter at the end is always fair, which will not favour any one of the paths. Even there exist bias, a simple solution of adding a delay in the structure can solve the problem. For example, if the arbiter favor the lower path,
by adding a delay to the upper path, it can have a head start. By looking at Figure \ref{fig:figure6}, it is clear that the CRPs will be exponential. Assuming there are n switching boxes, two possible cases in each switching box, so the 
number of CRPs is $2^{n}$, which indicate the arbiter PUF is a strong PUF. Arbiter PUF can also return a longer response by inputting K different challenges and getting a K bits response.

\begin{figure}[htp]
    \centering
    \includegraphics[width=12cm]{figures/figure6.jpg}
    \caption{Arbiter PUF structure}
    \label{fig:figure6}
    \end{figure}

The normal arbiter PUF is vulnerable to modeling attack, the purpose of XOR arbiter PUF is to increase the robustness of arbiter PUF. The basic concept is to integrate multiple parallel arbiter PUF, given the same challenge to each arbiter PUF and XORed each response to produce the final response(See Figure \ref{fig:figure7}) \cite{Reference5}.
According to simulation, book \cite{Reference4} provides the XOR arbiter PUF will have nonlinearity property that makes it harder to model.

\begin{figure}[htp]
    \centering
    \includegraphics[width=12cm]{figures/figure7.jpg}
    \caption{XOR arbiter PUF structure}
    \label{fig:figure7}
    \end{figure}

\section{Modeling attack on PUF}
Many different threats can perform on devices, such as eavesdropped, gaining access to the memory that stores secret keys. For PUF, the main threat is that attackers can use techniques 
like machine learning to simulate the behaviour of CRPs(so-called modeling), which means even without the devices, attackers can still respond correctly when a challenge is 
provided. Take arbiter puf as an example, assume an arbiter PUF with i switching box, and the challenge that applies to each switching box is c[i]. The two signals travel through the 
path determined by challenge, and arrive at the arbiter at different time because of the delay in each component. The final response depends on the sign of final delay 
difference:
\begin{equation}
    r =\begin{cases}
    1, & \text{if $\Delta c<0$}.\\
    0, & \text{otherwise}.
    \end{cases}
\end{equation}
which the delay difference is calculated by subtracting the upper path with the lower path's delay. The final delay difference $\Delta c$ can represent as $w^{T}\Phi$, where $w^{T}$ is 
a weight vector that represents delays for the components in PUF, and $\Phi$ is the applied i bits challenge \cite{Reference5}. $w^{T}\Phi = 0$ will provide a hyperplane that separates the space for $\Phi$, one side of the hyperplane are predicted as having response 1, the other side of the hyperplane
are predicted as having response -1. In conclusion, the correct hyperplane indicates a good prediction of PUF. Machine learning such as logistic regression can play the role well. The modeling result for an arbiter PUF with 64 switching boxes, 
by using the logistic regression can get a good performance of 99.9\% in a very short time with 18050 training CRPs \cite{Reference6}.

For the XOR arbiter PUF, it is also possible to use logistic regression to predict the behaviour but will be harder.

\section{Reconfigurability of PUF}
In order to alleviate the problem that PUF is vulnerable to modeling attack, reconfiguration property embedded on PUF has been proposed. For example, the one-time-PUF(so called 
OPUF), the general idea is that its configuration alters after every authentication session, which means CRPs behaviour of the PUF is changed and invalidate the modeling attack. This is based on the
forward unpredictability and backward unpredictability properties that OPUF possess \cite{Reference7}. Figure \ref{fig:figure8} shows the concept of forward unpredictability and backward unpredictability.
The orange line represents the forward unpredictability, which will ensure the responses collected before the reconfiguration process is invalid afterwards. As for the blue line, 
which represent backward unpredictability, it will ensure the pattern observed by an attacker using modeling attack on PUF' is invalid for predicting the PUF before reconfiguration process. 
In this case, assume an attacker is performing a modeling attack on the PUF, and required a certain amount of CRPs to gain proper prediction. However, the OPUF's structure keeps changing every execution, which means the CRPs 
behaviour is changing, so the modeling attack will can't be up to date or do not have time to collect enough CRPs.
\begin{figure}[htp]
    \centering
    \includegraphics[width=14cm]{figures/figure8.jpg}
    \caption{Concept of forward unpredictability and backward unpredictability}
    \label{fig:figure8}
    \end{figure}

DPUF can be considered as an example for creating the OPUF, it is built up with bit cells that contain capacitors and transistors, each cell stores information of values 0 and 1. However, these components will leak 
electrical signals every period of time, which means the behaviour might be eavesdropped \cite{Reference7}. Therefore performing the reconfiguration process every period of time is effective. The reconfigurability can be shown in Figure \ref{fig:figure9}, by varying parameters
such as refresh-pause interval and the memory block, where the former can cause random bit flip in the cells while the latter store the final response in a random memory block 
which will alter every period of time, PUF can present unpredictable behaviour that prevents from attacker perform modeling attack.

\begin{figure}[htp]
    \centering
    \includegraphics[width=10cm]{figures/figure9.jpg}
    \caption{Reconfigurability framework of DPUF}
    \label{fig:figure9}
    \end{figure}

\section{Summary}
In summary, chapter 2 describe PUF as a novel approach to increase the security robustness of electrical devices by making use of its unique physical characteristics and how it can be used
for devices authentication. Also, the working process and concepts for arbiter PUF and XOR arbiter PUF is introduced in detail. Last, the largest potential threat on PUF is called the modeling attack 
and its countermeasure are analyzed in detail.


